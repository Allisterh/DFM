\documentclass[a4paper]{article}

%% Language and font encodings
\usepackage[english]{babel}
\usepackage[utf8x]{inputenc}
% \usepackage[T1]{fontenc}

%% Sets page size and margins
\usepackage[a4paper,top=3cm,bottom=2cm,left=3cm,right=3cm,marginparwidth=1.75cm]{geometry}

%% Useful packages
\usepackage{amsmath}
\usepackage{graphicx}
\usepackage[colorlinks=true, allcolors=blue]{hyperref}
\usepackage{apacite} % APA style citations
\AtBeginDocument{\urlstyle{APACsame}}  % Links in APA citytions same formatting
\usepackage{tabulary}
\usepackage{natbib} % natbib citations: \citep{} and \citet{} for in-text
\usepackage{booktabs}
% If you use \tableofcontents, this adjusts the name
%\addto\captionsenglish{\renewcommand*\contentsname{Table of Contents}}


\title{\textbf{Dynamic Factor Models}\\ A Short Introduction}
\author{Sebastian Krantz}

\begin{document}
\maketitle

Dynamic factor models (DFM) postulate that a small number of unobserved "factors" can be used to explain a substantial portion of the variation and dynamics in a larger number of observed time series. In addition to producing estimates of the unobserved factors, dynamic factor models have many uses in forecasting and macroeconomic monitoring. One popular application for these models is "nowcasting", in which higher-frequency data is used to produce "nowcasts" of series that are only published at a lower frequency (Fulton, 2020). \newline

Dynamic Factor Models are set up in State-Space form and can be estimated using the Kalman Filter and a number of solution algorithms, the most popular one in the economics literature being the Expectation Maximization (EM) algorithm, due to it's robust numerical properties and the mixed frequency modification of \citet{banbura2014maximum}. \newline

\section{The Canonical (Exact) DFM}

A baseline dynamic factor model can be written as follows

\begin{align} \label{eq:do}
\textbf{x}_t &= \textbf{C}_0 \textbf{f}_t + \textbf{e}_t, \qquad\qquad\, \textbf{e}_t\sim N(\textbf{0}, \textbf{R}) \\ \label{eq:dt}
\textbf{f}_t &= \sum_{j=1}^p \textbf{A}_j \textbf{f}_{t-j} + \textbf{u}_t, \quad\ \textbf{u}_t\sim  N(\textbf{0}, \textbf{Q}_0),
\end{align}

where the unobserved factors $\textbf{f}_t$ are evolve according to a VAR(p) process and  \newline

\begin{tabular}{p{1cm}p{12cm}}
$\textbf{x}_t$ & $n \times 1$ vector of observed series at time $t$: $(x_{1t}, \dots, x_{nt})'$. Some observations can be missing.  \\\\
$\textbf{f}_t$ & $r \times 1$ vector of factors at time $t$: $(f_{1t}, \dots, f_{rt})'$.\\\\
$\textbf{C}_0$ & $n \times r$ measurement (observation) matrix.\\\\
$\textbf{A}_j$ & $r \times r$ state transition matrix at lag $j$. \\\\
$\textbf{Q}_0$ & $r \times r$ state covariance matrix.\\\\
$\textbf{R}$ & $n \times n$ measurement (observation) covariance matrix. It is diagonal by the assumption that all covariation between the series is explained by the factors $E[\textbf{x}_{it}|\textbf{x}_{-i,t},\textbf{x}_{i,t-1}, \dots, \textbf{f}_t, \textbf{f}_{t-1}, \dots] = \textbf{Cf}_t\ \forall i$.
\end{tabular}
\\\\

This model can be estimated using a classical form of the Kalman Filter and the Expectation Maximization (EM) algorithm, after transforming it to State-Space (stacked, VAR(1)) form:

\begin{align}
\textbf{x}_t &= \textbf{C} \textbf{F}_t + \textbf{e}_t, \qquad\quad \textbf{e}_t\sim N(\textbf{0}, \textbf{R}) \\
\textbf{F}_t &= \textbf{AF}_{t-1} + \textbf{u}_t, \quad\ \textbf{u}_t\sim  N(\textbf{0}, \textbf{Q}),
\end{align}
where $\textbf{x}_t$, $\textbf{e}_t$ and \textbf{R} are as in Eq. \ref{eq:do}, and the other matrices are

\begin{align}
\textbf{F}_{t\ (rp \times 1)}  &= (\textbf{f}_t', \textbf{f}_{t-1}', \dots, \textbf{f}_{t-p}')' = (f_{1t}, \dots, f_{rt}, f_{1,t-1}, \dots, f_{r,t-1}, \dots, f_{1,t-p}, \dots, f_{r,t-p})' \\[1em]
\textbf{C}_{(n \times rp)}  &= (\textbf{C}_0, \textbf{0}, \dots, \textbf{0}), \text{where \textbf{0} are $n\times r$ matrices of zeros for each factor lag} \\[1em]
\textbf{A}_{(rp \times rp)}  &= \begin{pmatrix}
\textbf{A}_1 & \textbf{A}_2 & \cdots & \textbf{A}_{p-1}  & \textbf{A}_p \\
\textbf{I}_1 & \textbf{0} & \cdots & \textbf{0} & \textbf{0} \\
\textbf{0} & \textbf{I}_2 & \cdots & \textbf{0} & \textbf{0} \\ 
\vdots & \vdots & \ddots & \vdots & \vdots \\
\textbf{0} & \textbf{0} & \cdots & \textbf{I}_{p-1} & \textbf{0}
\end{pmatrix}, \text{where \textbf{0}/\textbf{I} are $r\times r$ zero/identity matrices} \\[1em]
\textbf{u}_{t\ (rp \times 1)}  &= (\textbf{u}_t', \textbf{0}', \dots, \textbf{0}')', \text{with \textbf{0} a $r \times 1$ vector of zeros}  \\[1em]
\textbf{Q}_{(rp \times rp)}  &= \begin{pmatrix}
\textbf{Q}_0 & \textbf{0} & \cdots & \textbf{0} \\
\textbf{0} & \textbf{0}& \cdots & \textbf{0} \\ 
\vdots & \vdots & \ddots & \vdots  \\
\textbf{0} & \textbf{0} & \cdots & \textbf{0}
\end{pmatrix}, \text{where \textbf{0} are $r\times r$ zero matrices.} 
\end{align}

The estimation of this model using the EM algorithm, with initial values of the system matrices estimated through Principal Components Analysis to facilitate convergence for large $n$, is described in \citet{doz2012quasi}. This \emph{exact} DFM is quite restrictive as it assumes that all correlation in the data is explained by the unobserved common factors. In particular, it assumes:

\begin{enumerate}
\item Linearity and constant relationships (no structural breaks)
\item Idiosynchratic measurement (observation) errors (\textbf{R} is diagonal)
\item No direct relationship between series and lagged factors (can be relaxed)
\item No relationship between lagged error terms in the either measurement or transition equation (no serial correlation)
\end{enumerate}

Particularly assumption 4 is quite restrictive since it stipulates that all time dynamics in $\textbf{x}_t$ need to be accounted for by the factors. Within the framework of the \emph{exact} DFM, assumption 3 can easily be relaxed to allow $q$ lags of the factors in the measurement equation
\begin{equation}
\textbf{x}_t = \sum_{i=0}^q \textbf{C}_i \textbf{f}_{t-i} + \textbf{e}_t, \qquad \textbf{e}_t\sim N(\textbf{0}, \textbf{R}).
\end{equation}
In this case the stacked notation remains the same as long as $q < r-1$, with observation matrix 
\begin{equation}
\textbf{C}_{(n \times rp)}  = (\textbf{C}_0, \textbf{C}_1, \dots, \textbf{C}_q, \textbf{0}, \dots, \textbf{0})
\end{equation}
modified to estimate the lagged loadings $\textbf{C}_i$. However, because of the profileration of parameters, this extension has not received much attention. Furthermore, in the presence of significant lagged dynamics, increasing the number of factors $r$ is often successful in capturing these dynamics, with certain factors loading strongly on the lagged indicators and others on the contemporaneous ones. \newline


In the current practice of estimating large DFMs of economic time series, series for a given sector often have unmodeled sector-specific dynamics. Most of the economics literacture on DFMs has thus focused on relaxing restrictions about the error structure in factor models, introducing the notion of \emph{approximate} factor models \citep{stock2016dynamic}. A particular emphasis has been placed on relaxing assumption 4, i.e. allowing some of the dynamics of the time series to be unexplained by the common factors. 
% doz2020dynamic


\section{Approximate DFMs}

The most common form of \emph{approximate} DFM, introduced by \citet{chamberlain2983arbitrage}, allows for the observation errors $\textbf{e}_t$ to evolve according to an autoregressive AR(1) process
\begin{equation} \label{eq:ar1}
\textbf{e}_t = \mathbf{\Phi} \textbf{e}_{t-1} + \textbf{v}_t,\quad \textbf{v}_t\sim N(\textbf{0}, \textbf{R}),
\end{equation}
where $\mathbf{\Phi}$ is diagonal $n\times n$ with elements $\rho_i$. Following \citet{banbura2014maximum} this can be modelled as part of the state vector. 
%rewriting the dynamic form given by Equations \ref{eq:do} and \ref{eq:dt} as
% \textbf{x}_t = \textbf{C}_0^a \textbf{f}_t^a + \textbf{v}_t \quad \text{with} \quad \textbf{f}_t^a = (\textbf{f}_t', \textbf{e}_{t-1}')', \quad \textbf{C}_0^a = (\textbf{C}_0, \mathbf{\Phi}),\quad \textbf{v}_t\sim N(\textbf{0}, \textbf{R}),
%\begin{align}
%\textbf{x}_t &= \textbf{C}_0^a \textbf{f}_t^a \qquad\qquad\quad\ \ \text{with} \quad \textbf{f}_t^a = (\textbf{f}_t', \textbf{e}_t')', \quad \textbf{C}_0^a = (\textbf{C}_0, \textbf{I}) \\
%\textbf{f}_t^a &= \sum_{j=1}^p \textbf{A}_j^a \textbf{f}_{t-j}^a + \textbf{u}_t \quad \text{with} \quad \textbf{A}_1^a = (\textbf{A}_1, \mathbf{\Phi}),\quad \textbf{A}_{j>1}^a = (\textbf{A}_{j>1}, \textbf{0}), \quad \textbf{u}_t\sim  N(\textbf{0}, \textbf{Q}_0).
%\end{align}
In the stacked form, this model becomes 
\begin{align} \label{eq:doar1}
\textbf{x}_t &= \textbf{C}^a \textbf{F}_t^a \\
\textbf{F}_t^a &= \textbf{A}^a\textbf{F}^a_{t-1} + \textbf{u}_t^a, \quad\ \textbf{u}_t^a\sim  N(\textbf{0}, \textbf{Q}^a),
\end{align}
where $\textbf{x}_t$ is as in Eq. \ref{eq:do}, and the other matrices are

\begin{align}
\textbf{F}^a_{t\ (rp+n \times 1)}  &= (\textbf{f}_t', \textbf{f}_{t-1}', \dots, \textbf{f}_{t-p}', \textbf{e}_t')', \text{where $\textbf{e}_t$ is $n\times 1$ as in Eq. \ref{eq:ar1}} \\[1em]
\textbf{C}^a_{(n \times rp+n)}  &= (\textbf{C}_0, \textbf{0}, \dots, \textbf{0}, \textbf{I}), \text{where \textbf{0} is $n\times r$ and \textbf{I} is $n\times n$} \\[1em]
\textbf{A}^a_{(rp+n \times rp+n)}  &= \begin{pmatrix}
\textbf{A}_1 & \textbf{A}_2 & \cdots & \textbf{A}_{p-1}  & \textbf{A}_p & \textbf{0}\\
\textbf{I}_1 & \textbf{0} & \cdots & \textbf{0} & \textbf{0} & \textbf{0} \\
\textbf{0} & \textbf{I}_2 & \cdots & \textbf{0} & \textbf{0} & \textbf{0} \\ 
\vdots & \vdots & \ddots & \vdots & \vdots & \vdots \\
\textbf{0} & \textbf{0} & \cdots & \textbf{I}_{p-1} & \textbf{0} & \textbf{0} \\
\textbf{0} & \textbf{0} & \cdots & \textbf{0} & \textbf{0} & \mathbf{\Phi}
\end{pmatrix}, \text{where $\mathbf{\Phi}$ is $n\times n$ as in Eq. \ref{eq:ar1}} \\[1em]
\textbf{u}^a_{t\ (rp+n \times 1)}  &= (\textbf{u}_t', \textbf{0}', \dots, \textbf{0}', \textbf{v}_t')', \text{with $\textbf{v}_t$ $n \times 1$ as in Eq. \ref{eq:ar1}}  \\[1em]
\textbf{Q}^a_{(rp+n \times rp+n)}  &= \begin{pmatrix}
\textbf{Q}_0 & \textbf{0} & \cdots &  \textbf{0} \\
\textbf{0} & \textbf{0}& \cdots  & \textbf{0} \\ 
\vdots & \vdots & \ddots  & \vdots \\
\textbf{0} & \textbf{0} & \cdots & \textbf{R}
\end{pmatrix}, \text{where \textbf{R} is $n\times n$ as in Eq. \ref{eq:ar1}.} 
\end{align}

To still be able estimate this model using a classical form of the Kalman Filter, \citet{banbura2014maximum} introduce an error term $\mathbf{\epsilon} \sim N(\textbf{0}, \tilde{\textbf{R}})$ in Eq. \ref{eq:doar1}, with covariance $\tilde{\textbf{R}} = \kappa \textbf{I}$ for $\kappa$ a very small number. \newline

Like the \emph{exact} DFM, this model can also easily be extended to allow for $q$ factor lags in the observation equation, by estimating an observation matrix
\begin{equation}
\textbf{C}^a_{(n \times rp+n)}  = (\textbf{C}_0, \textbf{C}_1, \dots, \textbf{C}_q, \textbf{0}, \dots, \textbf{0}, \textbf{I}).
\end{equation}
The same comments given above about profileration of parameters and the possibility to include additional factors to account for lagged dynamics apply. \newline 

In general, \citet{doz2011two, doz2012quasi} show that in the presence of serial correlation, the factors are still consistently estimated as $n, T \to \infty$ using an \emph{exact} DFM specification. Thus modelling serial correlation is particularly important in smaller samples. \citet{banbura2014maximum} also show that modelling the serial correlation improves forecast accuracy, but only at short horizons.

\newpage 

\section{Mixed Frequency}

Building on the work of \citet{mariano2003new}, \citet{banbura2014maximum} and \citet{bok2018macroeconomic} (New York FED Nowcast) popularized EM based estimation of large mixed-frequency (monthly and quarterly) DFMs as workhorse models in economic nowcasting practice. The key addition of \citet{mariano2003new} was to model observed quarterly series by unobserved monthly counterparts, and then place appropriate restrictions on the observation matrix (\textbf{C}). \newline 

In particular, \citet{mariano2003new} consider the case of an observed quarterly series $X^q_t$ being the geometric mean of an unobserved monthly series $\tilde{X}^m_t$ and it's lags, i.e.
\begin{equation}
X^q_t = \tilde{X}^{m\frac{1}{3}}_t\tilde{X}^{m\frac{1}{3}}_{t-1}\tilde{X}^{m\frac{1}{3}}_{t-2},
\end{equation}
such that taking the natural log yields
\begin{equation}
\log(X^q_t) = \frac{1}{3}\log(\tilde{X}^m_t) + \frac{1}{3}\log(\tilde{X}^m_{t-1}) + \frac{1}{3}\log(\tilde{X}^m_{t-2}).
\end{equation}
Lagging this equation 3 times and subtracting it from itself yields the (approximate) quarterly growth rate of the quarterly series
\begin{equation}
\log(X^q_t) - \log(X^q_{t-3}) = \frac{1}{3}[\log(\tilde{X}^m_t) - \log(\tilde{X}^m_{t-3})] + \frac{1}{3}[\log(\tilde{X}^m_{t-1})-\log(\tilde{X}^m_{t-4})] + \frac{1}{3}[\log(\tilde{X}^m_{t-2})-\log(\tilde{X}^m_{t-5})].
\end{equation}
Adding and subtracting further lags and leads on the right hand side, and denoting the growth rate by lower case letters, i.e. $x^q_t = \log(X^q_t) - \log(X^q_{t-3})$ and $\tilde{x}^m_t = \log(\tilde{X}^m_t) - \log(\tilde{X}^m_{t-1})$, yields
\begin{align}
x^q_t &= \frac{1}{3}[\tilde{x}^m_t + \tilde{x}^m_{t-1} + \tilde{x}^m_{t-2}] + \frac{1}{3}[\tilde{x}^m_{t-1} + \tilde{x}^m_{t-2} + \tilde{x}^m_{t-3}] + \frac{1}{3}[\tilde{x}^m_{t-2} + \tilde{x}^m_{t-3} + \tilde{x}^m_{t-4}] \\
 &= \frac{1}{3}\tilde{x}^m_t + \frac{2}{3}\tilde{x}^m_{t-1} + \tilde{x}^m_{t-2} + \frac{2}{3}\tilde{x}^m_{t-3} + \frac{1}{3}\tilde{x}^m_{t-4}.
\end{align}
This is the result of \citet{mariano2003new}. \citet{banbura2014maximum} consider instead the quarterly series to be the product of the unobserved monthly series, i.e. starting from $X^q_t = \tilde{X}^m_t\tilde{X}^m_{t-1}\tilde{X}^m_{t-2}$, the final expression is
\begin{equation}
x^q_t = \tilde{x}^m_t + 2\tilde{x}^m_{t-1} + 3\tilde{x}^m_{t-2} + 2\tilde{x}^m_{t-3} + \tilde{x}^m_{t-4}.
\end{equation}
Since $\tilde{X}^{m\frac{1}{3}}_t$ (and thus $\tilde{x}^m_t$) is unobserved, this approach is equivalent, as only the relative weights on the lags of the unobserved series matter. 

\newpage

\bibliographystyle{apacite}
\bibliography{dynamic_factor_models} % This links to a file bibliography.bib with the citations

\end{document}